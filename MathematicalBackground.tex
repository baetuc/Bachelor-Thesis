\chapter{Mathematical Background}

The purpose of this chapter is to remind the reader basic notions regarding algebra and statistics, but also to analyze in detail concepts and algorithms concerning lattices, especially integer lattices. 

\section {Algebra}
The notions of group, cyclic group, ring and polynomial are considered to be previously known by the average reader. More information about the mentioned structures can be found in \cite{HPS08}, chapter 2. For a mathematical perspective over groups and polynomial rings properties, and also for a deep incursion in field extension theory, the reader can explore \cite{BB15}.

\begin{enumerate}
	\item \textbf{Ideals and Quotient Rings.} 
	
	\textbf{Definition 5.} \textit{Let $(R, +, \cdot)$ be a commutative finite ring. An \textbf{ideal} of $R$ is a nonempty set $\ci \subseteq R$ that is closed under addition and $ax=xa \in \ci, \forall x \in \ci, \forall a \in R.$} \\
	
	Let $(R, +, \cdot)$ be a finite ring and $\ci$ be an ideal of $R$. Also, let $"\sim"$ be an equivalence relationship on $R$, defined by $x \sim y \iff \exists a\in \ci $ such that $x = y + a$. The equivalence class of an element $x \in R$ is usually noted with $\hat{x}$, and it represents the set $\{y \in R : y \sim x\}$. The equivalence classes generate a partition of the set $R$, named \textit{quotient set}, and denoted by $R/\ci$. It is known that $(R/\ci, +, \cdot)$ is also a ring, and it is called the \textbf{quotient ring} $R/\ci$.
	
	\textbf{Remark 3.} \textit{An example of a quotient ring is the well-known ring $\bZ_p = \bZ/p\bZ.$}
	
	\item \textbf{Cyclotomic polynomials.}
	
	\textbf{Definition 6 \cite{BB15}.} \textit{Let $n \geq 1$ be an integer and $P_n $ be the set of all the $n^{th}$ primitive roots of unity. Then, the \textbf{$n^{th}$ cyclotomic polynomial} is $\Phi_n = \displaystyle{\prod_{\xi \in P_n}}(X-\xi)$}.
	
	\textbf{Remark 4.} Using the fact that $\Phi_{p^k}(X) = \Phi_p(X^{p^{k-1}})$, for any positive integers $k, p$, with $p-$prime, it can be proved that $\Phi_{2^k}(X) = X^{2^{k-1}} + 1,$ for any integer $k \geq 1$.
	
	\item \textbf{Vector spaces}.
	
	Throughout the paper, vectors and matrices are thickened, e.g. $\textbf{v}$. Also, every vector space is considered to be contained in $R^m$, with $m \geq 1$, integer.\\
	
	\textbf{Definition 7 \cite{HPS08}.} \textit{Let $m$ be a positive integer. A \textbf{vector space} $V$ is a subset of $\bR^m$, such that for every $\alpha_1, \alpha_2 \in \bR$ and every} $\textbf{v}_1, \textbf{v}_2 \in V$, it holds: $\alpha_1 \vv_1 + \alpha_2 \vv_2\in V$.
	
	The user is expected to master the concepts of \textit{linear combination, linear independence, basis, vector orthogonality, basis orthogonality}. For a quick review over the mentioned concepts, visit \cite{HPS08}.\\
	
	The only algorithm regarding vector spaces to be presented in the current paper is the  \textbf{Gram-Schmidt Algorithm}. It receives as input a basis $\{\vv_1, .., \vv_n\}$ of the vector space $V$ and outputs $\{\vv_1^*,..,\vv_n^*\}$ - an orthonormal basis of $V$. The algorithm is presented below:

\begin{tcolorbox}[colframe=black,colback=white,arc=0pt,outer arc=0pt]
	\begin{algorithmic}[1]
		\State Set $\vv_1^* = \vv_1$
		\For{$i \la 2$ \textbf{to} $n$}
		\State Compute $\mu_{ij} = \vv_i \cdot \vv_j^*$ $/$ $ ||\vv_j^*||$,  for $1 \leq j < i$
		\State Set $\vv_i^* = \vv_i - \displaystyle{\sum_{j = 1}^{i-1}} \mu_{ij} \vv_j^* $
		\EndFor
	\end{algorithmic}
\end{tcolorbox}

	Intuitively, $\mu_{ij}$ represents the length of the projection of $\vv_i$ over $\vv_j^*$. Therefore, the substraction $\vv_i - \mu_{ij} \vv_j^*$ generate the projection of $\vv_i$ over the orthogonal complement of $\vv_j^*$, which leads to the desired output. 
\end{enumerate}	

\section{Lattices}

