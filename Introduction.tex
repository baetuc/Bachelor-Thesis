\chapter*{Introduction}


\space Usually, cryptographic primitives are constructed under the assumption that several problems are intractable, i.e. there exist no polynomial-running time algorithm to solve them. Vercauteren \cite{Ver13} realized an extensive research regarding the intractable problems that are most used in cryptography, such as integer factoring, discrete logarithm, computational Diffie-Hellman, shortest vector problem and many others.\\

In the last ten years, bilinear maps proved to be very useful in cryptography. Making use of the interesting properties of such maps, cryptographers managed to construct schemes for one-round three-party key exchange \cite{Jou00}, identity based encryption \cite{BoF01} and many other applications. After the moment that bilinear maps proved to be undoubtedly  useful, researchers have tried to generalize the concept. Thus, multilinear maps were defined and the search for their applications has begun. Boneh and Silverberg \cite{BoS02} showed that symmetric multilinear maps can be used to realize a one-round multi-party key exchange scheme but, after their attempts to construct such maps failed, they drew the conclusion that "such maps might have to either come from outside the realm of algebraic geometry, or occur as 'unnatural' computable maps arising from geometry."\\

Garg, Gentry and Halevi \cite{GGH13} proposed a construction based on lattices that approximate the multilinear maps in hard-discrete-logarithm groups. Using this candidate, they could construct an application to multipartite Diffie-Hellman key exchange scheme and also the first construction of Attribute-Based Encryption for general circuits \cite{GGH+13a}. A short period after the aforementioned candidate was proposed, Coron, Lepoint and Tibouchi \cite{CLT13} created a similar construction, based on integers instead of lattices.\\

However, these construction proved to be susceptible to attacks, and a devastating zeroizing attack for the integer construction is presented thoroughly in \cite{CKC+15}. Numerous fixing tentatives of these schemes were designed, but for each of them there was found at least another attack. Therefore currently, new methods of constructing multilinear maps and Graded Encoding Schemes still constitute an open field, very interesting for cryptographers. 
\newpage

\section*{Contribution}

This work represents a survey over the bilinear and multilinear maps and their applications. Furthermore, in this paper is presented the concept of Graded Encoding Scheme, a modality of approximating multilinear maps. The core of the paper is represented by the review of the lattice-based construction, designed by Garg, Gentry and Halevi in \cite{GGH13}. 

\section*{Organization}

The paper is divised into 5 sections. First one - introduction bla-bla. In the second section, bla bla bla ... etc etc.