\chapter{Proposed Encoding Scheme}

Garg, Gentry and Halevi constructed a Graded Encoding Scheme based on ideal lattices, in \cite{GGH13}. Their construction was the first candidate to approximate multilinear maps, therefore had a tremendous impact in the world of cryptography. The current chapter intends to present the system designed by the three aforementioned authors, in an in-depth manner. 

\section{Overview}

To start with, the ring $R$ denotes the cyclotomic polynomial ring $\bZ[X]/(X^n +1)$, for some integer $n$ - power of two. The ring $R$ is often considered to be the lattice $\bZ^n$, because a correspondence between the two structures is obvious. Also, $g \in R$ is a short ring element and $\id = \langle g \rangle$ is the principal ideal generated by $g$. Additionaly, an element $z$ is extracted randomly in $R_q$. \\

The elements to be encoded by the scheme are the equivalence classes (or \textbf{cosets}) of the quotient ring $QR = R / \id$, denoted by $\eh$, for some $e \in R$. \\

A level-zero encoding of a coset $\eh$ is a short vector in that coset. The existence of a small vector in any coset is assured by the fact that $g$ is a short element, therefore the basis $B(g) = \{g, Xg,...,X^{n-1}g\}$ has all elements small - only circular permutations of the vector $g$, eventually with a changed sign. Hence, the fundamental domain itself of $B(g)$ is small and because for any $e \in R$, there exists an element $e_g \in \mathcal{F}(B(g))$ such that $e_g \in \eh$, the result follows immediately.\\

For any $i \in \{1,2...,k\}$, the set of all level-$i$ encodings of a coset $\eh$ is $S_i^{\eh} = \{\frac{c}{z^i}\in R_q : c \in eh, ||c|| < q^{1/8}\}$. The value $||c||$ will be referred to as the \textbf{noise level} of the encoding.\\

\section{Efficient Procedures}

The procedures to be presented are a specific case of the efficient procedures introduced in Subsection 1.4.1. Thus, in this section only the implementation of the mentioned functions will be exposed, as in \cite{GGH13}:

\begin{enumerate}
	\item \textbf{Instance generation}
\end{enumerate}