\chapter{Multilinear Maps and Graded Encoding Systems}

In this chapter, multilinear maps are defined and also, the particular case of bilinear maps is discussed, along with results concerning bilinear maps over groups of composite order. Thereafter, \textit{Graded Encoding Schemes} are defined, as an approximate to multilinear maps. \\

\textit{Observation:} Regarding the multilinear maps and Graded Encoding Systems schemes, and also for the lattice-based candidate designed in \cite{GGH13}, the paper encompasses one subsection of efficient procedures, and another one of hardness assumptions. The reader should be aware of this detail and realize the analogy and differences of the mentioned schemes.


\section{Bilinear Maps}
As stated before, bilinear maps are a specific case of multilinear maps. They proved to be a highly useful tool in cryptography, with many applications, such as: tripartite protocol \cite{Jou00}, identity based encryption \cite{BoF01} and Attribute-based encryption scheme for monotone boolean formulas \cite{TiD14}. In this section, bilinear maps are only defined, while next section presents a relationship between self-bilinear maps and multilinear maps. \\

\textbf{Definition 1.} (Bilinear Map \cite{BCM16}). \textit{Given the cyclic groups $G$ and $G_t$ (written additively) of the same order $p$, a (symmetric) map $e:G \cart G \ra G_t$ is said to be bilinear if the following properties hold:
\begin{enumerate}
	\item \textbf{(Bi-linearity)}\space  $e({g_1}^{x_1}, {g_2}^{x_2}) = e(g_1,g_2)^{x_1x_2},$ for any $x_1,x_2 \in \bZ_p$ and any $g_1, g_2 \in G$;
	\item  \textbf{(Non-degeneracy)} \space If $g_1, g_2\in G$ are generators of $G$, then $e(g_1, g_2)$ is a generator of $G_t$;
	\item \textbf{(Efficient computability)} There exists a polynomially-bounded algorithm to compute $e(g_1,g_2)$, for any $g_1,g_2 \in G$.
\end{enumerate} 
}

\section{Cryptographic Multilinear Maps}
\textbf{Definition 2.} (Multilinear Maps \cite{Rot13}). \textit{ Let $k \geq 2$ be an integer number and $G_1, G_2,...,G_k, G_T$ be $k + 1$ cyclic groups (written additively), of same order $p$. Then, a $k-$multilinear map is a mapping $e:G_1\cart...\cart G_k \ra G_T$, with the following properties:
	\begin{enumerate}
		\item \textbf{(Linearity)} For every $g_1\in G_1, ..., g_k \in G_k$, every $i\in \{1,2,..,k\}$ and every $\alpha\in \bZ_p$, it holds that:
\begin{align*}
			e(g_1,...,\alpha \cdot g_i, ..,g_k) = \alpha \cdot e(g_1,...,g_k))
\end{align*}
		\item \textbf{(Non-degeneracy)} If $g_1\in G_1, ..., g_k \in G_k$ are generators of their respective groups, then $e(g_1,...,g_k)$ is a generator of $G_T$.
	\end{enumerate}
}

\subsection{From Self-Bilinear to Multilinear Maps}

\textbf{Definition 3.} \textit{A self-bilinear map is a bilinear map where the domain and target groups are the same.} \\

\textbf{Proposition 1.} \textit{Let $G$ be a cyclic group of order $p$ and $e:G\cart G\ra G$ be a self-bilinear map. Therefore, a $k-$multilinear map $e_k:G^k \ra G$ can be constructed from $e$, for any $k \geq 2$.}\\

\textbf{\textit{Proof.}} The proof is realized by induction. First, for the base case $k = 2$, it is trivial to observe that $e$ itself is a 2-liniar map. Then, suppose that an $n-$multilinear map $e_n:G^n\ra G$ can be constructed starting from $e$, and it can be easily shown that a $(n+1)$-multilinear map $e_{n+1}:G^{n+1}\ra G$ can be constructed, as follows:
\begin{align*}
	e_{n+1}(g_1,..,g_n, g_{n+1}) = e(e_n(g_1,.., g_n), g_{n+1}), \forall g_1, .., g_{n+1} \in G.
\end{align*}
Indeed, from the fact that $e_n$ is multilinear it follows that, for any $g_1\in G_1, ..., g_n \in G_n$, any $i\in \{1,..,n\}$ and any $\alpha \in \bZ_p$,  $e_n(g_1,...,\alpha \cdot g_i, ..,g_n) = \alpha \cdot e_n(g_1,...,g_n))$. Using the bilinearity of $e$, it results that $e_{n+1}$ respects the \textbf{linearity} condition.\\
Let $g_1, ...,g_n$ be generators of $G$. Then, using the fact that $e_{n}$ is $n-$multilinear, it follows that $e_n(g_1,..,g_n)$ is also a generator of $G$. Corroborating the last result with the non-degeneracy property of $e$, it ensues that $e_{n+1}$ respects \textbf{non-degeneracy} condition, from which the conclusion that $e_{n+1}$ is a $(n+1)$-multilinear map can be drawn. \qed\\







