\chapter {Security}

In order to realize an implementation of the GES scheme presented at \textbf{Chapter 3}, it must be proved to be secure, at least in a theoretical manner. The security study started even from the original paper of Garg, Gentry and Halevi, where the weaknesses of Discrete Logarithm on level smaller than $k$, along with averaging attacks are discussed.\\

Furthermore, after the publication of the paper, there followed a sum of attacks on the scheme, that would break some security assumptions, such as \textit{Subgroup Membership} ($SubM$) or \textit{Decision Linear} ($DLIN$). The initial ones, known as "zeroizing" attacks, used the public level-one encodings of zero in order to crack some assumptions. Then, extended weak Discrete Logarithm attacks were set up even in constructions without public encryptions of zero. One of the latest results over the security of the scheme is represented by the \textit{annihilation attacks}, exposed in \cite{MSZ16}.\\

As depicted in the previous chapter, one of the applications of the scheme is \textit{indistinguishability obfuscation}. One of the main advantages of iO is that they do not require the existence of public encodings of zero. Because most of the attacks over \textit{GGH13} scheme are based on the presence of the aforementioned public representations of zero, it follows that iO schemes remain secure.\\

However, a new kind of attack was presented in 2016 by Miles, Sahai and Zhandry in \cite{MSZ16}. They conceived the first polynomial-time cryptanalysis of candidate iO schemes over \textit{GGH13}, using \textit{annihilation attacks}. Therefore, even the future of iO implementations may become dark, and further analysis and fixes are required. \\

The first attack to be presented in the current chapter is a "zeroizing" one, followed closely by attacks that could be mounted in the assumption that some elements could be found. Finally, the principle of \textit{averaging attacks} is overviewed, as an ending of the chapter and as the paper as well.

\section{A zeroizing attack}

The "zeroizing" attack is also called \textit{weak discrete logarithm attack}. It was discovered soon after the publication of GGH construction \cite{GGH13}, and it lead to the break of $SubM$ and $DLIN$ problems, considered to be secure in the initial paper. As no similar attack was designed for CLT13 scheme \cite{CLT13} very fast, the analogue system designed by Coron, Lepoint and Tibouchi was thought to provide secure $DLIN$ and $SubM$ instantiation. However, in \cite{CKC+15} an attack was designed for the CLT scheme, so powerful that lead to a total break, i.e. all private parameters could be efficiently recovered.\\

\begin{tcolorbox}[colframe=black,colback=white,arc=0pt,outer arc=0pt]
	\begin{center}
		\textbf{Zeroizing attack (Weak Discrete Logarithm)}
	\end{center}
	\begin{algorithmic}[1]
		\For {$i$ from 1 to $l$}
		\State Given $u_i = \big[ \frac{\textbf{d}_i}{\zz^{s_i}} \big]_q$, with $1 \leq s_i < k$, compute $\textbf{f}_i = [u_i \cdot \xx_j \cdot$\pzt$\cdot \textbf{y}^{k - s_i - 1}]_q$.
		\EndFor
		\State Using $\textbf{f}_1, \textbf{f}_2, ..., \textbf{f}_l$, compute a basis of $\id$.
	\end{algorithmic}
\end{tcolorbox}
~\\
The zeroizing attack is based on the existence of public encodings of zero, namely $\xx_1,..., \xx_m$ and the public encoding of one, $\textbf{y}$. Particularly, for any fixed $j \in \{1,2,...,m\}$, consider the level-one representation of zero, $\xx_j = \big[ \frac{\textbf{b}_j}{\zz} \big]_q$, with $\textbf{b}_j \in \hat{0}$. Because $\textbf{b}_j \in \hat{0}$, therefore there exists a short $c_j$ such that $\textbf{b}_j = c_j \cdot \textbf{g}$. Thus, $\xx_j = \big[ \frac{c_j \cdot \textbf{g}}{\zz} \big]_q$.\\

Then, if the attacker may obtain valid representations of elements on levels lower than $k$ (and for each encoding, he knows the specific level), he may succeed in his attack. Suppose the attacker knows $l$ valid elements, $u_i = \big[ \frac{\textbf{d}}{\zz^{s_i}} \big]_q, \forall i \in \{1,2,...,l\}$. Then, for each element, he can compute, as exemplified in \cite{Gar15}:

\begin{center}
	$\textbf{f}_i = [u_i \cdot \xx_j \cdot$\pzt$\cdot \textbf{y}^{k - s_i - 1}]_q = \big[ \frac{\textbf{d}_i}{\zz^i} \cdot \frac{c_j \cdot \textbf{g}}{\zz} \cdot \frac{\textbf{h}\zz^k}{\textbf{g}} \cdot \frac{\textbf{a}^{k- i - 1}}{\zz^{k-i-1}}\big]_q=$
	
\end{center}
	
\begin{center}
	$= [\textbf{d}_i \cdot c_j \cdot \textbf{h} \cdot \textbf{a}^{k-i-1}]_q \stackrel{\mathclap{\normalfont\mbox{\scriptsize{(1)}}}}{=} \textbf{d}_i \cdot c_j \cdot \textbf{h} \cdot \textbf{a}^{k-i-1} \equiv (\textbf{d}_i \cdot \underbrace{c_j \cdot \textbf{h} }_{\Delta_j} )   (\mod \id)$.

\end{center}

The equality (1) holds because the value: $\textbf{d}_i \cdot c_j \cdot \textbf{h} \cdot \textbf{a}^{k-i-1}$ is significantly lower than $q$, therefore the element reduced modulo $q$ has the same representation as the element without the reduction. The vigilant reader may note that $\Delta_j$ is invariant of $\xx_j$. Thus from a level-$s_i$ encoding of $\textbf{d}_i$ represented by $u_i$, an attacker could compute the value $\textbf{f}_i \in \widehat{\textbf{d}_i \cdot\Delta_j}$.\\

Consider that $||\Delta_j|| \approx ||\textbf{h} || =  O(\sqrt{q})$, therefore $||\textbf{f}_i||> q^{\frac{1}{2}}$ with overwhelming probability. Thus $\textbf{f}_i $ is not upper-bounded by the correct value in order to represent a valid level-zero encoding of an element.\\

Hence, applying this procedure for all the public $\xx_j$'s, the attacker will have at discretion many elements from the ideal $\langle \textbf{h} \rangle$. From all the elements recovered, he may recover a basis of the principal ideal lattice $\langle \textbf{h} \rangle$. Therefore, it will be fairly easy to construct a basis for the fractional principal ideal $\big\langle\frac{1}{\textbf{h}} \big \rangle$ in $\mathbb{K}$.\\

Using a similar algorithm as the one presented above, the attacker may recover a basis for the principal ideals $\langle\textbf{h} \cdot \textbf{g} \rangle$ and $\langle \textbf{h} \cdot \textbf{a} \rangle$. Utilizing the found basis for $\big\langle\frac{1}{\textbf{h}} \big \rangle$, then a basis for $\langle \textbf{a} \rangle$ and $\langle \textbf{g} \rangle = \id$ may be easily found.\\

The conclusion of the current subsection is that the ideal $\id$ cannot remain entirely hidden, but also that finding a small basis for it is still hard, therefore the construction is viable, as its complete break is realized by finding the small generator $\textbf{g}$ of $\id$.\\



It must be noted that the zeroizing attack does not affect \textbf{GDDH} assumption, as the level of the known elements must be less than $k$ in order to take part of the algorithm. However, it is obvious that the presence of the public level-one encodings of zero represent a vulnerability of the construction. Therefore, in order to be able to ensure $SubM$ and $DLIN$ security, the system must provide different methods to randomize the encoding procedure at level one, without the usage of the aforementioned public parameters.

\section{Attacks with known elements}

As the title suggests, the section will evaluate the amount of damage that could be realized to the security of the construction, in the assumption that several particular elements could be available to the attacker.\\

The analysis is performed regarding the \textbf{GDDH} problem, with the known parameters of the scheme, but also with the parameters bound to the \textbf{GDDH} instance, which are similar to the output of \textbf{genGDDH} algorithm:

\begin{itemize}
	\item $u_i = \big[ \frac{e_i}{\zz} \big]_q$, for $i \in \{0,1,...,k\}$ - $k+1$ level-one encodings of random elements, with $||e_i|| < 2^\lambda \sigma \sqrt{n}, \forall i \in \{0,1,...,k\}$;
	
	\item $w = \big[ \frac{c}{\zz^k} \big]_q$ - the challenge element, which represents the level$-k$ encoding of $\displaystyle{\prod_{i = 0}^{k} e_i}$ or of a random coset of $R/\id$.
\end{itemize}

\subsection {A short element of the ideal $\id$}


\begin{tcolorbox}[colframe=black,colback=white,arc=0pt,outer arc=0pt]
	\begin{center}
		\textbf{Attack with a known short element of $\id$}\\
		(Suppose that the known element is $\textbf{dg}$)
	\end{center}
	\begin{algorithmic}[1]
		\State Set $\textbf{p}_{zt}' = [\textbf{dg} \cdot \textbf{p}_{zt}]_q$.
		\Statex
		\State Set $v_1 = [\textbf{p}_{zt}' \cdot w]_q$.
		\Statex
		\State Set $v_2 = \Bigg[ \textbf{p}_{zt}' \cdot \displaystyle{\prod_{i = 1}^{k} u_i} \Bigg]_q$.
		\Statex
		\State Set $e^{sup}_0 = v_1 \text{ div } v_2 (\text{mod } \id)$. \Comment Element to be compared to $e_0$
		\Statex
		\State Compute $u_0^{sup} = e_0^{sup} \cdot \textbf{y}$.	 \Comment Level-one encoding of $e_0^{sup}$
		\Statex
		\State \textbf{Output:} \textbf{isZero}(params, \pzt, $u_0 - u_0^{sup}$). 
	\end{algorithmic}
\end{tcolorbox}
~\\

The flow of the algorithm is fairly natural, as it can be observed from the step-by-step explanation above. The proof of correctness follows immediately. \\

First, suppose that $w$ is a level$-k$ encoding of $\displaystyle{\prod_{i = 0}^{k} e_i}$. Thus, there exists a vector $c$ such that $w = \Big[ \frac{c \textbf{g} + \prod_{i = 0}^{k} e_i}{\zz^k} \Big]_q.$ Let the short element in $\id$ be $\textbf{dg}$, with $\textbf{d}$ a small vector. Then, a modified zero-test parameter will be computed, $\textbf{p}_{zt}' = [\textbf{dg} \cdot \textbf{p}_{zt}]_q = [\textbf{d} \cdot \textbf{h} \cdot \zz^k]_q$. Multiplying $\textbf{p}_{zt}'$ by both $w$ and $\displaystyle{\prod_{i = 1}^{k} u_i}$, it yields:

\begin{center}
	$
	\begin{cases}
			v_1 = [\textbf{p}_{zt}' \cdot w]_q \\
			 v_2 = \bigg[ \textbf{p}_{zt}' \cdot \displaystyle{\prod_{i = 1}^{k} u_i} \bigg]_q
	\end{cases}
	 \implies 
		\begin{cases}
	v_1 = \textbf{d} \cdot \textbf{h} \cdot \bigg( c\textbf{g} + \displaystyle{\prod_{i = 0}^{k} e_i} \bigg) \\
	v_2 = \textbf{d} \cdot \textbf{h} \cdot \displaystyle{\prod_{i = 1}^{k} e_i}
	\end{cases} (1).
	$
	
\end{center}

Next, it can be easily seen that the quantity $e_0^{sup} = v_1 \text{ div } v_2\text{ (mod }\id)$ must be congruent with $e_0 \text{ (mod }\id)$. In order to calculate $e_0^{sup}$, one must compute the Hermite Normal Form (HNF) of the lattice, using the basis found using the zeroizing attack. Knowing that the HNF basis has the first element of the main diagonal equal to $N(\id)$, the attacker could easily find the norm of the ideal.\\

Let $\mu_1 = v_1 \text{ (mod }N(\id))$ and let $\mu_2 = v_2 \text{ (mod }N(\id))$. Because $N(\id) \in \bZ$, then $\mu_1, \mu_2 \in \bZ$. Also $N(I) \in \id$, thus:
\begin{center}
 $\mu_1 \equiv v_1 \text{ (mod }\id) $ and $\mu_2 \equiv v_2 \text{ (mod }\id)$. (2)
 \end{center}

Using relations (1) and (2) and the fact that $\id = \langle \textbf{g} \rangle$, it follows immediately that: 

\begin{center}
	$
	\begin{cases}
	\mu_1 \equiv v_1 \equiv \textbf{d} \cdot \textbf{h} \cdot \displaystyle{\prod_{i = 0}^{k} e_i}  \text{ (mod }\id)  \\
	\mu_2 \equiv v_2 \equiv \textbf{d} \cdot \textbf{h} \cdot \displaystyle{\prod_{i = 1}^{k} e_i}  \text{ (mod }\id)
	\end{cases}.
	$
\end{center}

Using the relations above, it can be easily seen that $\mu_2 \cdot e_0 \equiv \textbf{d} \cdot \textbf{h} \cdot e_0 \cdot \displaystyle{\prod_{i = 1}^{k} e_i}  \text{ (mod }\id)$, therefore:

\begin{center}
	$\mu_2 \cdot e_0 \equiv \mu_1 \text{ (mod }\id) $ (3).
\end{center}

Supposing that $\mu_2 \in \bZ_{N(\id)}^*$, let $\eta = \mu_1 \cdot \mu_2^{-1} \text{ (mod }N(\id))$. Then, $\eta \cdot \mu_2 \equiv \mu_1 \text{ (mod }N(\id))$ and knowing that $N(\id) \in \id$, it follows that: 
\begin{center}
	$\mu_2 \cdot \eta \equiv \mu_1 \text{ (mod }\id)$ (4).
\end{center}

Using relations (3) and (4) and the fact that $\mu_2$ is coprime with $N(\id)$, it results that $\eta \equiv e_0 \text{ (mod }\id)$. Reducing $\eta$ modulo the short rotation basis of $\textbf{dg}$, which is obviously small, because $\textbf{dg}$ itself is small, it results a short element $e_0^{sup} \equiv e_0 \text{ (mod }\id)$. \\

Therefore, an attacker could recover a valid level-zero representation of the coset $\widehat{e_0}$, and immediately compute the level-one representation of it, $u_0^{sup} = e_0^{sup} \cdot \textbf{y}$. Then, the result of \textbf{isZero}($u_0 - u_0^{sup}$) is returned, which is \textbf{true}, in this case. In the case that $w$ is not a level$-k$ representation of the product $\displaystyle{\prod_{i = 0}^{k}e_i}$, the same argument as above may be applied to observe that $e_0^{sup} \notin \widehat{e_0}$, therefore the procedure \textbf{isZero} will return \textbf{false}. Thus, the algorithm is correct and it may be mounted by an attacker in order to solve the \textbf{GDDH} problem in the GGH13 construction.

\subsection{A small multiple of $\frac{1}{h}$}

\begin{tcolorbox}[colframe=black,colback=white,arc=0pt,outer arc=0pt]
	\begin{center}
		\textbf{Attack with a known short multiple of $\frac{1}{\textbf{h}}$}\\
		(Suppose that the known element is $\vv = \frac{\textbf{d}}{\textbf{h}}$)
	\end{center}
	\begin{algorithmic}[1]
		\State Set $\textbf{p}_{zt}' = [\vv \cdot \textbf{p}_{zt}]_q$.
		\Statex
		\State Set $\textbf{p}_{zt}'' = [(\textbf{p}_{zt}')^2 \cdot \xx_0]_q$. \Comment Zero-test parameter for level $2k-1$
		\Statex
		\State Set $w' = [w \cdot \textbf{y}^{k-1}]_q$. \Comment Lift $w$ with $k-1$ levels, therefore $w'$ is on level $2k-1$
		\Statex
		\State Set $p' = \bigg[ \displaystyle{\textbf{y}^{k-2} \cdot \prod_{i = 0}^{k} u_i} \bigg]$. \Comment Lift $p$ with $k-2$ levels, therefore $p'$ is on level $2k-1$
		\Statex
		\State \textbf{Output:} \textbf{isZero}(params, $\textbf{p}_{zt}'', w' - p'$). \Comment Test for zero on level $2k-1$
	\end{algorithmic}
\end{tcolorbox}
~\\

An ingenuous attacker may note that the availability of a zero-test parameter on any level higher than $k$ would lead to a simple, yet devastating attack. Suppose that $\textbf{p}_{zt}'$ is a level$-k'$ zero test parameter, with $k' > k.$ Then, the attacker may lift the challenge element $w$ to level $k'$ by multiplying it with $\textbf{y}^{k' - k}$. Soon after, he may also lift the product $\displaystyle{\prod_{i = 0}^{k} u_i}$ to level $k'$, by multiplying it with $\textbf{y}^{k' - k -1}$ (note that the condition $k' > k$ is essential). Therefore, the attacker knows a level$-k'$ representation of the desired product, but also a level$-k'$ representation of the challenge element. Thus, he may use the zero-test parameter on level $k'$ to decide whether or not the two elements encode the same coset.\\

It can be shown that a level$-(2k)$ zero-test parameter can be constructed from a small multiple of $\frac{1}{\textbf{h}}$, as presented in \cite{GGH13}. Let $\vv = \frac{\textbf{d}}{\textbf{h}}$ be the small multiple of $\frac{1}{\textbf{h}}$. The attacker may compute the quantity $\textbf{p}_{zt}' = [\vv \cdot \textbf{p}_{zt}]_q = \Big[\frac{\textbf{d}\zz^k}{\textbf{g}} \Big]_q$ and afterwards $\textbf{p}_{zt}'' = [(\textbf{p}_{zt}')^2 \cdot \xx_0]_q = \Big[\frac{\textbf{d}^2 \textbf{z}^{2k}}{\textbf{g}^2} \cdot \frac{\textbf{b}_0 \textbf{g}}{\textbf{z}} \Big]_q = \Big[ \frac{(\textbf{d}^2\textbf{b}_0) \cdot \zz^{2k-1}}{\textbf{g}} \Big]_q$. Therefore, reminding that a level$-k$ zero test parameter is some $\textbf{p}_{zt} = \Big[ \frac{\textbf{hz}^k}{\textbf{g}} \Big]_q$, it follows that the constructed $\textbf{p}_{zt}''$ is a valid level$-(2k-1)$ zero-test parameter.\\

Thus, an attacker that knows a short multiple of $\frac{1}{\textbf{h}}$ may construct a level$-(2k-1)$ zero-test parameter and then mount an attack that solves \textbf{GDDH} in polynomial time.\\

\textbf{Remark 14.} \textit{The algorithm is correct for $k \geq 2$. However, the case $k < 2$ is impractical, and even $k = 2$ refers to bilinear maps, which may be constructed in a different way (e.g. using Weil-Tate pairings).}\\

\textbf{Remark 15.} \textit{The paper \cite{GGH13} contains a \textbf{mistake} in Section 6.3.3, by considering $\xx_0 = [\textbf{b}_0 \textbf{g}]_q$, which is not the way the element is constructed. In the current paper, the mistake is rectified, therefore the zero-test parameter is valid for level $(2k-1)$ and not for level $2k$, as stated in \cite{GGH13}.}

\subsection{A small multiple of $hg^r$}

\begin{tcolorbox}[colframe=black,colback=white,arc=0pt,outer arc=0pt]
	\begin{center}
		\textbf{Attack with a known short multiple of $\textbf{hg}^r$}\\
		(Suppose that the known element is $\vv = \textbf{d} \cdot \textbf{hg}^r$)
	\end{center}
	\begin{algorithmic}[1]
		\State Set $\vv' = \sqrt[r]{\vv}$. \Comment An approximation of \textbf{g}
		\Statex
		\State Recover $\textbf{g}$ from $\vv'$.
		\Statex
		\State \textbf{Output:} Found $\textbf{g}$.
	\end{algorithmic}
\end{tcolorbox}
~\\

Suppose that that attacker knows the quantity $\vv = \textbf{d} \cdot \textbf{hg}^r$, for a large value of $r$. Because $\textbf{dh}$ is independent of $\textbf{g}^r$ and fairly random, then the value $\sqrt[r]{\textbf{dh}}$ tends to a constant, as $r$ increases (to infinity).\\

 Therefore, a large value for $r$ may imply that $\vv'$ represents a reasonable approximation of $\textbf{g}$. Using the algorithm used for solving the Closest Principal Ideal Generator problem, presented in \cite{Gar15}, Section 9.5, the attacker may recover the generator $\textbf{g}$, thus breaking the entire scheme.\\
 
 However, the attack is somewhat implausible for the GGH13 construction, because in practice $n$ is very large compared to $k$, hence the condition that $r$ is much larger than $k$ is not sufficient (taking into consideration the bounds presented at \textbf{Lemma 2}). It is presented though, because there exists a small risk that an attacker may implement an efficient version of the algorithm.
 
 \section{Averaging attacks}
 
 Averaging attacks refer to algorithms that are capable to compute the invariant $\textbf{h}$, by observing many elements that have the structure $\textbf{h} \cdot \textbf{r}$, with a random $\textbf{r}$. Otherwise, after seeing many elements, they may return the "common" part of all the input encodings. \\
 
 Averaging attacks may be used together with the zeroizing attack. It is reminded that during the aforementioned attack, numerous elements that have the form $\alpha \cdot \textbf{g}$ are available to the attacker. Therefore, if she could mount an averaging attack using the found encodings, the construction would be completely broken.\\
 
 Garg presents in his PhD thesis, \cite{Gar15}, Chapter 9, four versions of averaging attacks. The first one is used to recover the quantity $\textbf{h} \cdot \bar{\textbf{h}}$, where $\textbf{h}$ is the invariant of the encodings. Then, a way to recover $\textbf{h}$ from $\textbf{h} \cdot \bar{\textbf{h}}$, using the Gentry-Szydlo algorithm is exposed. The last two attacks are extensions to the previous ones, designed by Nguyen-Regev and Ducas-Nguyen. \\ 
 
 Luckily, in practice there is not a sufficient number of public level-one encodings of zero such that the mentioned attack to succeed. However, it is important that the awareness of the existence of such attacks is raised, and to already find methods to defend against this type of attack.
 